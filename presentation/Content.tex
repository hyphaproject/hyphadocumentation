\part[Presentation]{Presentation}
\section{Introduction}

\begin{frame}
	\frametitle{Introduction}
	Hypha is a framework for homeautomation.
	
	It contains of following parts.
	\begin{itemize} 
		\item Hypha library
		\item Runner
		\item Manager
		\item Webmanager
		\item Plugins
		\item Handlers
	\end{itemize}
\end{frame}

\begin{frame}
	\frametitle{Hypha Library}
	Hypha library contains core of the framework and is used in runner, managers, handlers and plugins.
	\begin{itemize}
		\item c++11
		\item Boost
		\item Poco
	\end{itemize}
\end{frame}

\begin{frame}
	\frametitle{Runner}
	\begin{itemize}
		\item executes handlers
		\item executes plugins
		\item sends data to other runner
		\item use one runner instance per device
	\end{itemize}
\end{frame}

\begin{frame}
	\frametitle{Runner}
	\begin{itemize}
		\item c++11
		\item python
		\item Boost
		\item Poco
		\item OpenCV
	\end{itemize}
\end{frame}

\begin{frame}
	\frametitle{Plugin}
	\begin{itemize}
		\item Sensor/Actor Module
	\end{itemize}
\end{frame}

\begin{frame}
	\frametitle{Handler}
	\begin{itemize}
		\item Logic / Handle Modules
	\end{itemize}
\end{frame}

\begin{frame}
	\frametitle{Manager}
	\begin{itemize}
		\item configurates device connection
		\item configurates plugins
		\item configurates handlers
	\end{itemize}
\end{frame}

\begin{frame}
	\frametitle{Manager}
	\begin{itemize}
		\item c++11
		\item Qt
		\item Boost
		\item Poco
	\end{itemize}
\end{frame}

\section{Hardware}
\begin{frame}
	\frametitle{Hardware}
	\begin{itemize} 
		\item Raspberry PI
		\item Arduino
	\end{itemize}
\end{frame}

\section{Features}
\begin{frame}
	\frametitle{Software}
	\begin{itemize}
		\item Configuration Database based (MySQL / SQLite)
		\item HTTP based Communication
		\item JSON based Protocol
		\item DNS based Network
	\end{itemize}
\end{frame}

\begin{frame}
	\frametitle{Data Cache}
		Pointer like system that allows to send uuid as pointer and to get data later.
		
		Data can be found at same host or on other host in the network.
\end{frame}

\section{TODO}
\begin{frame}
	\frametitle{TODO}
	\begin{itemize}
		\item Declare Protocol
		\item HTTPS Communication
		\item DNS to P2P/Certificate
		\item Machine Learning
		\item Big Data
	\end{itemize}
\end{frame}



