\documentclass[german,a4paper,11pt]{report}

\usepackage[german]{babel}
\usepackage[utf8]{inputenc}
\usepackage{hyperref}
\usepackage{xcolor}
\usepackage{listings}
\lstset{basicstyle=\ttfamily,
	showstringspaces=false,
	commentstyle=\color{red},
	keywordstyle=\color{blue}
}



% Title Page
\title{Hypha Admin Manual}
\author{Christian M}


\begin{document}
\maketitle

\begin{abstract}
	This manual is about how to build and install hypha on a RaspberryPI.
\end{abstract}
\tableofcontents 
\chapter{Hardware}
You need the following:
\begin{itemize}
	\item RaspberryPI 2 B
	\item 8GB micro SD
	\item USB-Mouse
	\item USB-Keyboard
	\item USB-Wifi Stick
\end{itemize}

\chapter{Raspberry PI}

\section{Betriebssystem installieren}
First of all you have to download an operating system for the RaspberryPI.
You will find information and downloads on following site: \url{https://www.raspberrypi.org/downloads/}

This manual will work with Raspbian as operating system.

\section{Copy SD Card}
Follow the instructions on this site \url{https://www.raspberrypi.org/documentation/installation/installing-images/README.md}
\begin{lstlisting}[language=bash,caption={copy sd card}]
sudo fdisk -l
sudo dd if=/dev/sdd of=~/raspberry-pi.img
sudo dd if=~/raspberry-pi.img of=/dev/sdd
sync
\end{lstlisting}



\section{First Setup}
\subsection{hostname}
Using command
\begin{lstlisting}[language=bash,caption={raspi-config}]
raspi-config
\end{lstlisting}
you can change the hostname for example to 'hypha1'.

\subsection{WIFI setup}

\begin{lstlisting}[language=bash,caption={interfaces}]
sudo nano /etc/network/interfaces
\end{lstlisting}
Write following in the file.

\begin{lstlisting}[language=bash,caption={/etc/network/interfaces}]
auto lo
iface lo inet loopback
iface eth0 inet dhcp

auto wlan0
allow-hotplug wlan0
iface wlan0 inet dhcp
wpa-ap-scan 1
wpa-scan-ssid 1
wpa-ssid "YOUR-WIFI-NAME"
wpa-psk "YOUR-WIFI-PASSWORD"
\end{lstlisting}

Restart wifi.
\begin{lstlisting}[language=bash,caption={restart interfaces}]
sudo service networking restart
\end{lstlisting}

\subsection{Apt Sources}
Install 'sudo apt-get install apt-transport-https'
to access ssl secured repositories.

\subsection{Backports}
We need some packages that are not in the standard repository or too old.

You have to write following into the \texttt{/etc/apt/sources.list} file.

\begin{lstlisting}[language=bash,caption={/etc/apt/sources.list}]
...
deb http://ftp.de.debian.org/debian/ jessie-backports main
\end{lstlisting}

\subsection{remove X11}
For hypha we don't need any desktop environment or X-server.
\begin{lstlisting}[language=bash,caption={remove x11}]
sudo aptitude remove --purge x11-common
sudo aptitude autoremove
\end{lstlisting}


\subsection{Update}

\begin{lstlisting}[language=bash,caption={update}]
sudo aptitude update
sudo aptitude upgrade
\end{lstlisting}

\section{additional packages}
\subsection{Git, CMake}
For downloading and building hypha you have to install git and cmake first.

\begin{lstlisting}[language=bash,caption={install git, cmake}]
sudo aptitude install git cmake
\end{lstlisting}

\subsection{MySQL}
For data storage we use mysql as database.

\begin{lstlisting}[language=bash,caption={install mysql}]
sudo aptitude install mysql-server mysql-client
\end{lstlisting}

\subsection{QT5 Framework}
We need qt in version 5, not 4. Qt5 can be found in the backports.
\begin{lstlisting}[language=bash,caption={install qt5}]
sudo aptitude install qt5-default \
libqt5sql5-mysql libqt5sql5-odbc libqt5serialport5-dev
\end{lstlisting}

\subsection{Boost Libraries}
\begin{lstlisting}[language=bash,caption={install boost}]
sudo aptitude install libboost-dev \
	libboost-thread-dev libboost-signals-dev \
	libboost-filesystem-dev libboost-system-dev \
	libboost-test-dev
\end{lstlisting}

\subsection{Poco Libraries}
\begin{lstlisting}[language=bash,caption={install poco}]
sudo aptitude install libpoco-dev
\end{lstlisting}

\subsection{OpenCV Libraries}
There are also dependencies for the opencv libraries.
\begin{lstlisting}[language=bash,caption={install opencv}]
sudo aptitude install libopencv-dev libxvidcore-dev libavformat-dev\
	libx264-dev libmjpegtools-dev
\end{lstlisting}

\subsection{python}
There are plugins running python code.
For that we need the python runtime and pip to install some python packages.
\begin{lstlisting}[language=bash,caption={install python, python-pip}]
sudo aptitude install python-dev python-pip
\end{lstlisting}

\subsection{espeak}
Espeak is for the speaker plugin.
\begin{lstlisting}[language=bash,caption={install libespeak}]
sudo aptitude install espeak
\end{lstlisting}

\subsection{ldap}
Ldap is an option as user database.
\begin{lstlisting}[language=bash,caption={install libldap}]
sudo aptitude install libldap2-dev
\end{lstlisting}

\subsection{nxp-explore}
To use the exploreNFC reader we need an library from github.
\begin{lstlisting}[language=bash,caption={compile install nxp-explore}]
git clone https://github.com/svvitale/nxppy.git
sudo aptitude install python-setuptools
cd nxppy
sudo python setup.py build install
\end{lstlisting}

\subsection{WiringPi}
To access GPIO pins of the RaspberryPI directly we need the library WiringPI.
\begin{lstlisting}[language=bash,caption={install wiringpi}]
git clone git://git.drogon.net/wiringPi
cd wiringPi
./build
\end{lstlisting}

\chapter{Database}
\section{Installation}
First we install the MySQL server.
\begin{lstlisting}[language=bash,caption={mysql-server}]
sudo aptitude install mysql-server
\end{lstlisting}
Then we allow other computer in the network to access the server.
You have to change the file \texttt{/etc/mysql/my.cnf} and comment out following:
\begin{lstlisting}[language=bash,caption={/etc/mysql/my.cnf}]
#bind-address           = 127.0.0.1
\end{lstlisting}

In mysql command line we create a new user for remote connection.

\begin{lstlisting}[language=bash,caption={login to mysql}]
mysql -u root -p
\end{lstlisting}
\begin{lstlisting}[language=sql,caption={create mysql user}]
CREATE USER 'pi'@'%' IDENTIFIED BY 'raspberry';
\end{lstlisting}
\section{Create UserDatabase}
\begin{lstlisting}[language=sql,caption={create userdatabase}]
CREATE DATABASE `hypha\_userdb` DEFAULT CHARACTER SET utf8;
GRANT ALL PRIVILEGES ON `hypha\_userdb`.* TO 'pi'@'%'
IDENTIFIED BY 'raspberry';
\end{lstlisting}
\section{Create Database}
\begin{lstlisting}[language=sql,caption={create database}]
CREATE DATABASE `hypha\_db` DEFAULT CHARACTER SET utf8;
GRANT ALL PRIVILEGES ON `hypha\_db`.* TO 'pi'@'%'
IDENTIFIED BY 'raspberry';
\end{lstlisting}

\chapter{Additional configurations}
\section{cronjob}
To run hypharunner at reboot change your crontab.
\begin{lstlisting}[language=bash]
sudo crontab -e
\end{lstlisting}
\begin{lstlisting}[language=bash,caption={crontab}]
@reboot cd /home/pi/hypharunner/build/bin && ./hypharunner
\end{lstlisting}

\section{Watchdog}
To reboot the raspberry when it crashes install watchdog.
\begin{lstlisting}[language=bash,caption={watchdog}]
sudo apt-get install watchdog
echo "bcm2708_wdog" | sudo tee -a /etc/modules
\end{lstlisting}
Commend out following in the watchdog config file \texttt{/etc/watchdog.conf}
\begin{lstlisting}[language=bash,caption={/etc/watchdog.conf}]
watchdog-device		= /dev/watchdog
max-load-1			= 24
\end{lstlisting}
Add the watchdog daemon also as cronjob.
\begin{lstlisting}[language=bash]
sudo crontab -e
\end{lstlisting}
\begin{lstlisting}[language=bash,caption={crontab}]
@reboot sleep 60 && sudo service watchdog start
\end{lstlisting}

\section{KadNode}
KadNode is a P2P Service to resolve public keys to network ip addresses.

\subsection{Installation}
We have do download sources, create and install kadnode package.

\begin{lstlisting}[language=bash,caption={install kadnode}]
sudo apt-get install build-essential debhelper devscripts hardening-includes
sudo apt-get install libnatpmp-dev libminiupnpc-dev libsodium-dev
git clone https://github.com/mwarning/KadNode
cd KadNode
dpkg-buildpackage
sudo dpkg -i kadnode.*.deb
\end{lstlisting}

\subsection{Key Pair}

Now we have to create a public and private key pair.
\begin{lstlisting}[language=bash,caption={kadnode key pair}]
kadnode --auth-gen-keys
\end{lstlisting}

\subsection{kadnode.conf}

Edit file /etc/kadnode/kadnode.conf and add
\begin{lstlisting}[language=bash,caption={/etc/kadnode/kadnode.conf}]
--value-id <secret key>
\end{lstlisting}
After that restart the service:
\begin{lstlisting}[language=bash,caption={}]
/etc/init.d/kadnode restart
\end{lstlisting}

\chapter{Webcam Überwachung}
\section{Installation}
\section{RaspiCAM}
Das Kernelmodul für die RaspiCAM muss bei jedem Reboot gestartet werden.
Dafür fügen wir der Datei /etc/modules ein 'bcm2835-v4l2' an.

\subsection{Networkfolder}
If you have a windows (samba) network folder you can connect to it as following:
\begin{lstlisting}[language=bash,caption={}]
sudo aptitude install -y cifs-utils
sudo mkdir -p /mnt/shared/webcam
sudo echo "//IP/webcam/ \
/mnt/shared/webcam cifs uid=0,gid=0,rw,username=x,passwd=y 0 0"\
 >> /etc/fstab
 sudo crontab -e
\end{lstlisting}
\begin{lstlisting}[language=bash,caption={cronjob}]
@reboot	sleep 20 && mount -a
\end{lstlisting}

\chapter{Troubleshootings}


\end{document}
