\documentclass[english,a4paper,11pt]{report}

\usepackage[english]{babel}
\usepackage[utf8]{inputenc}
\usepackage{hyperref}
\usepackage{xcolor}
\usepackage{listings}
\lstset{basicstyle=\ttfamily,
	showstringspaces=false,
	commentstyle=\color{red},
	keywordstyle=\color{blue}
}



% Title Page
\title{Hypha Developer Manual}
\author{Christian M}


\begin{document}
\maketitle

\begin{abstract}
	This Manual describes how to build the hypha project repositories.
\end{abstract}

\chapter{URLs}

\section{Website}
The Website is hosted on 
\href{https://hyphaproject.github.io/}{hyphaproject.github.io}

\section{Chat}
There is a public chat for the project at 
\href{https://gitter.im/hyphaproject/public}{gitter.im}

\section{Git}
Our primary Version Control System is GIT.
Following repositories exist on github.com:
\begin{itemize}
	\item https://github.com/hyphaproject/hypha
	\item https://github.com/hyphaproject/hyphahandlers
	\item https://github.com/hyphaproject/hyphaplugins
	\item https://github.com/hyphaproject/hyphamanager
	\item https://github.com/hyphaproject/hypharunner
\end{itemize}

\chapter{Prerequirements}
Install git, cmake, ssl and some databases
\begin{lstlisting}[language=bash,caption={prerequirements}]
sudo apt-get install git cmake rpm
sudo apt-get install openssl libssl-dev
sudo apt-get install libmysqlclient-dev libpq-dev\
 libiodbc2 libiodbc2-dev
\end{lstlisting}


\chapter{Development}

\section{Qt}
\begin{lstlisting}[language=bash,caption={install qt libraries}]
sudo apt-get install qt5-default qttools5-dev\
 libqt5sql5-mysql libqt5sql5-odbc libqt5serialport5-dev
\end{lstlisting}

\section{Poco}
We need to clone and Build the Poco Libraries from Github.
\begin{lstlisting}[language=bash,caption={git clone poco}]
git clone --depth 1 https://github.com/pocoproject/poco
\end{lstlisting}
\begin{lstlisting}[language=bash,caption={build poco}]
sudo sh build_cmake.sh
\end{lstlisting}


\section{Boost}
\begin{lstlisting}[language=bash,caption={install boost libraries}]
sudo apt-get install libboost-all-dev
\end{lstlisting}

\section{OpenCV}
OpenCV is used in some plugins.
\begin{lstlisting}[language=bash,caption={install opencv libraries}]
sudo apt-get install libopencv-dev
\end{lstlisting}

\section{ConfDesc}
Confdesc is a library to describe data types of configuration values.
Confdesc is used in HyphaManager and HyphaWebManager.
\begin{lstlisting}[language=bash,caption={git clone confdesc}]
git clone --recursive https://github.com/FalseCAM/confdesc
\end{lstlisting}
\begin{lstlisting}[language=bash,caption={build poco}]
cd confdesc
mkdir -p build
cd build
cmake ..
make -j3
\end{lstlisting}

\chapter{hypha}
\section{hypha}
Hypha is a library used by runner and managers.
\begin{lstlisting}[language=bash,caption={git clone hypha}]
git clone --recursive https://github.com/hyphaproject/hypha
\end{lstlisting}
Change to cloned directory and build with cmake
\begin{lstlisting}[language=bash,caption={build hypha}]
cd hypha
mkdir -p build
cd build
cmake ..
make -j3
\end{lstlisting}

\section{hyphahandler}
Hyphahandler contains the logic plugins.
\begin{lstlisting}[language=bash,caption={git clone hyphahandlers}]
git clone --recursive https://github.com/hyphaproject/hyphahandlers
\end{lstlisting}
Change to cloned directory and build with cmake
\begin{lstlisting}[language=bash,caption={build hyphahandlers}]
cd hyphahandlers
mkdir -p build
cd build
cmake -Dhypha_DIR=../hypha ..
make -j3
\end{lstlisting}

\section{hyphaplugins}
Hyphaplugins contains the input-output plugins.
\begin{lstlisting}[language=bash,caption={git clone hyphaplugins}]
git clone --recursive https://github.com/hyphaproject/hyphaplugins
\end{lstlisting}
Change to cloned directory and build with cmake
\begin{lstlisting}[language=bash,caption={build hyphaplugins}]
mkdir -p build
cd build
cmake -Dhypha_DIR=../hypha ..
make -j3
\end{lstlisting}

\section{hypharunner}
Hypharunner is a service to execute and connect the handlers and the plugins.
\begin{lstlisting}[language=bash,caption={git clone hypharunner}]
git clone --recursive https://github.com/hyphaproject/hypharunner
\end{lstlisting}
Change to cloned directory and build with cmake
\begin{lstlisting}[language=bash,caption={build hypharunner}]
cd hypharunner
mkdir -p build
cd build
cmake -Dhypha_DIR=../hypha ..
make -j3
\end{lstlisting}

\section{hyphamanager}
Hyphamanger is a desktop application used to configure connected modules.
\begin{lstlisting}[language=bash,caption={git clone hyphamanager}]
git clone --recursive https://github.com/hyphaproject/hyphamanager
\end{lstlisting}
Change to cloned directory and build with cmake
\begin{lstlisting}[language=bash,caption={build hyphamanager}]
cd hyphamanager
mkdir -p build
cd build
cmake -Dhypha_DIR=../hypha ..
make -j3
\end{lstlisting}

\section{hyphawebmanager}
Hyphamanger is a web server used to configure connected modules.
\begin{lstlisting}[language=bash,caption={git clone hyphawebmanager}]
git clone --recursive https://github.com/hyphaproject/hyphawebmanager
\end{lstlisting}
Change to cloned directory and build with cmake
\begin{lstlisting}[language=bash,caption={git clone hyphawebmanager}]
cd hyphawebmanager
mkdir -p build
cd build
cmake -Dhypha_DIR=../hypha ..
make -j3
\end{lstlisting}


\chapter{Project in detail}

\section{Handler}
Handler are modules that contain the logic.
They are there own project and will be compiled into a plug-in library.
They will be loaded dynamically by hypharunner and the managers.
Handlers configuration will be written to database table 'handler'.
They need a unique ID, a type, a hostname where they will run on and a json configuration string.


\section{IO Plugins}

\subsection{espeak}

\begin{center}
	\begin{tabular}{| l | l | l | l |}
		\hline
		Parameter & Type & IO & Summary \\ \hline
		say & String & Output & Speaks a text over speaker. \\
		\hline
	\end{tabular}
\end{center}

\subsection{explorenfc}
RFID cardreader for RaspberryPi.

\begin{center}
	\begin{tabular}{| l | l | l | l |}
		\hline
		Parameter & Type & IO & Summary \\ \hline
		device & String & Input & Id string of a rfid card. \\
		\hline
	\end{tabular}
\end{center}

\subsection{fingerprintzfm}
Fingerprint by Adafruit.
\begin{center}
	\begin{tabular}{| l | l | l | l |}
		\hline
		Parameter & Type & IO & Summary \\ \hline
		authuser & String & Input & Username of recognized persons fingerprint \\
		state & String & Output & State to change to. \\
		\hline
	\end{tabular}
\end{center}

\subsection{rfid}
RFID cardreader for Arduino.

\begin{center}
	\begin{tabular}{| l | l | l | l |}
		\hline
		Parameter & Type & IO & Summary \\ \hline
		device & String & Input & ID of an rfid card. \\
		\hline
	\end{tabular}
\end{center}

\subsection{rpigpio}
An output plugin to control the Raspberry PI GPIOs.
It's possible to control LEDs with this plugin.

\begin{center}
	\begin{tabular}{| l | l | l | l |}
		\hline
		Parameter & Type & IO & Summary \\ \hline
		red & Boolean & Output & Red LED on/off \\
		green & Boolean & Output & Green LED on/off \\
		yellow & Boolean & Output & Yellow LED on/off \\
		beep & Integer & Output & Signal with frequency \\
		\hline
	\end{tabular}
\end{center}

\subsection{wifi}
Plugin to scan MAC Addresses connected to the Wifi network.

\begin{center}
	\begin{tabular}{| l | l | l | l |}
		\hline
		Parameter & Type & IO & Summary \\ \hline
		device & String & Input & MAC Addresses of scanned Wifi devices \\
		\hline
	\end{tabular}
\end{center}

\chapter{Tools}
\section{clang-format}
You can always write "make clang-format" in build folder to format the source files with google style guide.

\end{document}